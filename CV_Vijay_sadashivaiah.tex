
\documentclass{resume}
\usepackage[utf8]{inputenc}
\usepackage{hyperref}
\usepackage{tabularx}
\usepackage{booktabs}
\usepackage{textcomp}
\usepackage{fancyhdr}
\usepackage[normalem]{ulem}
\usepackage{enumitem}

\newlist{publications}{enumerate}{1}
\setlist[publications, 1]{label = P\arabic*}

\newlist{presentations}{enumerate}{1}
\setlist[presentations, 1]{label = C\arabic*}

\pagestyle{fancy}
\fancyhf{}
\renewcommand{\headrulewidth}{0pt}
%\rhead{Vijay Sadashivaiah}
\newcommand{\tabitem}{~~\llap{\textbullet}~~}
\usepackage[left=0.6in,top=0.6in,right=0.6in,bottom=0.6in]{geometry}
\name{Vijay Sadashivaiah}
\address{+1 443 447 3694 \\ \href{mailto:vjs@jhu.edu}{vjs@jhu.edu} \\ \href{https://vjysd.github.io}{https://vjysd.github.io}}
\address{4002 B Linkwood Road, Baltimore, MD 21210}
%\def\nameskip{\bigskip}
\def\sectionskip{\medskip}
\lhead{\textit{Vijay Sadashivaiah}}

\begin{document}
\thispagestyle{empty}
  \begin{rSection}{Education}
\begin{tabular*}{\textwidth}{@{\extracolsep{\fill}}lr@{}}
\textbf{Johns Hopkins University, Whiting School of Engineering} & Baltimore, MD \\
{Master of Science in Biomedical Engineering}, GPA: 3.87/4.00 & May 2017 \\
Thesis: {``Towards Pain Control by Modelling the Interactions in a Mammalian Nerve Fiber''}\\ \\
\textbf{Visvesvaraya Technological University, PES Institute of Technology} & Bengaluru, India \\
{Bachelor of Engineering in Electronics and Communication Engineering}, GPA: 9.32/10.00 & May 2015 \\
Visiting student at \textbf{Massachusetts Institute of Technology}, Cambridge, MA & June - September 2014 \\
Thesis: {``Transient Imaging: Seeing the unseen''}\\
\end{tabular*}
  \end{rSection}

\vspace{1em}

        \begin{rSection}{Awards and Fellowships}
\begin{tabular*}{\textwidth}{@{\extracolsep{\fill}}lr@{}}
\textbf{Recipient}, Biomedical Engineering Departmental Fellowship - Johns Hopkins University & 2015 - 2017\\ 
\textbf{Semi-finalist}, Data Incubator Challenge - The Data Incubator & 2017\\
\textbf{Recipient}, Foundation Leenaards' Summer Research Fellowship - EPFL &  2015\\
\textbf{Recipient}, University Merit Scholarship - PES Institute of Technology & 2011 - 2015 \\ 
\textbf{Recipient}, ``Code Something that Matters" Scholarship - Vecna Robotics & 2015\\
\textbf{Global Finalist}, Vertech City Challenge - Vertech Symposium  & 2014\\
\textbf{Winner}, Best Student Project - IEEE International Conference on Impact of E-Technology on US & 2014\\
\textbf{Global Finalist}, Intel Global Challenge -  UC Berkeley & 2013 \\
\textbf{Finalist}, Biotechnology Entrepreneurship Student Teams (BEST) - Department of Biotechnology, India & 2013\\
\textbf{Global Semi-finalist}, Go Green in the City - Schneider Electric & 2013 \\
\textbf{Winner}, Best Project Award - Innovation for a Better Tomorrow (IBETO)  & 2013\\
\end{tabular*}
    \end{rSection}

  \vspace{1em}

  \begin{rSection}{Research Experience}
\begin{rSubsection}{Lieber Institute for Brain Development, \textit{Research Associate}}{August 2017 - Present}{Adviser: Dr. Qiang Chen, \textit{Data Science/Computational Biology}}{Baltimore, MD}
\item Exploring novel data driven methods to analyze imaging genetics data from developmental brain disorders
\item Building supervised learning models to identify underlying biological pathways in Schizophrenia
\item Developed models are based on Deep Neural Network (CNN's) and Support Vector Machine frameworks
\item Presented preliminary results at local and international scientific meetings [P1, P2, C1, C2]
\item \uline{Technical Skills:} Python, R, SQL, Tensorflow, Keras, SPM, Linux
    \end{rSubsection}

    \begin{rSubsection}{Johns Hopkins University, \textit{Research Assistant}}{September 2015 - May 2017}{Adviser: Dr. Sridevi V. Sarma, \textit{Neuromedical Control Systems Lab}}{Baltimore, MD}
\item Spearheaded collaboration between 3 principal investigators for thesis work
\item Constructed probabilistic, functional \& mechanistic models of mammalian nerve fiber using mathematical models
\item Quantified the interactions in a nerve fiber to test the performance of electrical nerve stimulation
\item Optimized our codebase by 70\% and storage by 60\% by developing efficient NEURON scripts
\item Amalgamated the findings into a journal article [P4]
\item \uline{Technical Skills:} MATLAB, NEURON, Unix, Linux
    \end{rSubsection}
  
    \begin{rSubsection}{École Polytechnique Fédérale de Lausanne, \textit{Summer Researcher}}{June 2015 - August 2015}{Adviser: Dr. Carl Petersen, \textit{Laboratory of Sensory Processing}}{Lausanne, Switzerland}
\item Studied the neural circuits involved in goal directed sensorimotor interactions
\item Analyzed over 1.5 TB of voltage sensitive dye images across multiple trials
\item Developed an interactive graphical platform to visualize neuroimagaing data on MATLAB
\item Co-authored a peer reviewed journal article [P7]
\item \uline{Technical Skills:} MATLAB, Python, Igor Pro, Linux
    \end{rSubsection}
\newpage
    \begin{rSubsection}{Massachusetts Institute of Technology, \textit{Summer Researcher}}{June 2014 - September 2014}{Adviser: Dr. Ramesh Raskar, \textit{Camera Culture Lab}}{Cambridge, MA}
\item Designed a high speed imaging system to capture light in motion (Bachelor's thesis)
\item Improved the depth resolution of conventional imaging system using multi-frequency light sources
\item Authored a do it yourself manual for the imagaing system
\item Featured on MIT website and BBC news
\item \uline{Technical Skills:} Verilog, MATLAB, C, Linux, Circuit design, Optics
    \end{rSubsection}
    
 \begin{rSubsection}{PES Institute of Technology, \textit{Undergraduate Researcher}}{June 2012 - May 2014}{Adviser: Dr. Srinivas A, \textit{Healthcare Innovation Lab}}{Bengaluru, India}
\item Collaborated with local and international hospitals to analyze real world clinical data
\item Used signal processing techniques learnt in class to analyze human physiology data
\item Worked on time series analysis of EKG, Skin Conductance, ERG etc
\item Presented results at international technical conferences and competitions [C6]
\item \uline{Technical Skills:} Verilog, MATLAB, Rapid prototyping, Circuit design, Arduino, Raspberry Pi, Sensors
 \end{rSubsection}
  \end{rSection}

\vspace{1em} 

\begin{rSection}{Publications}
\begin{publications}
\item \textbf{Sadashivaiah, V.}, Goldman, A., Ulrich, B., Radulescu, E., Breman, K. F., Mattay, V. S., Weinberger, D. R., Chen, Q.; Using machine learning to identify novel neuroimaging phenotypes associated with cognitive dysfunction in Schizophrenia. (in preparation)
 
\item \textbf{Sadashivaiah, V.}, Goldman, A., Ulrich, B., Straub, R. E., Calliott, J. H., Breman, K. F., Mattay, V. S., Weinberger, D. R., Chen, Q.; Exploring Shared Brain Cognitive Networks and the Related Genetic Components using Three-way Parallel ICA. (in preparation) 

\item Ren, M., Chen, Q., \textbf{Sadashivaiah, V.}, Li, Y., Zhu, S., Mezeivtch, K., Hu, Z.,  Qin, LS L., Li, X., Tian, Q., Parades, D., Zhu, J., Wang, K. H., Weinberger, D. R., Yang, F.; Abnormal hippocampal-mPFC synchrony in the KCNH2-3.1 transgenic mouse model. (in preparation)

\item \textbf{Sadashivaiah V.}, Sacre P., Guan Y., Anderson W. S., Sarma S. V.; Modeling the interactions between stimulation and physiologically induced APs in a mammalian nerve fiber: dependence on frequency and fiber diameter. (in review)

\item \textbf{Sadashivaiah, V.}, Sacré, P., Guan, Y., Anderson, W. S., Sarma, S. V.; Studying the Interactions in a Mammalian Nerve Fiber: A Functional Modeling Approach, 40th Annual International Conference of the IEEE Engineering in Medicine \& Biology Society, Honalulu, Hawaii, 2018. (in press)

\item \textbf{Sadashivaiah, V.}, Sacré, P., Guan, Y., Anderson, W. S., Sarma, S. V.; Selective Relay of Afferent Sensory Induced Action Potentials from Peripheral Nerve to Brain and the Effects of Electrical Stimulation, 40th Annual International Conference of the IEEE Engineering in Medicine \& Biology Society, Honalulu, Hawaii, 2018. (in press)

\item Kyriakatos A., \textbf{Sadashivaiah V}., Zhang Y., Motta A., Auffret M., Petersen C. H.; Voltage-sensitive dye imaging of mouse neocortex during a whisker detection task, Neurophotonics.

\item \textbf{Sadashivaiah, V.}, Sacré, P., Guan, Y., Anderson, W. S., Sarma, S. V.; Electrical neurostimulation of a mammalian nerve fibers: A probabilistic versus mechanistic approach, 39th Annual International Conference of the IEEE Engineering in Medicine \& Biology Society, Jeju Island, South Korea, 2017.

\item Gunnarsdottir, K., \textbf{Sadashivaiah, V.}, Kerr, M., Santaniello, S., Sarma, S. V.; Using Demographic and Time Series Physiological Features to Classify Sepsis in the Intensive Care Unit, 38th Annual International Conference of the IEEE Engineering in Medicine \& Biology Society, Florida, 2016.

\item Das, A., Swedish, T., Wahi, A., Moufarrej, M., Noland, M., Gurry, T., Michel, E. M., Aksel, D., Wagh, S., \textbf{Sadashivaiah, V.}, Zhang, X., Raskar, R.; Mobile phone based mini-spectrometer for rapid screening of skin cancer. Proc. SPIE 9482, Next-Generation Spectroscopic Technologies VIII, 94820M (June 3, 2015).

\end{publications}
\end{rSection} 

\begin{rSection}{Presentations}
\begin{presentations}
\item \textbf{Sadashivaiah, V.}, Goldman, A., Ulrich, B., Radulescu, E., Breman, K. F., Mattay, V. S., Weinberger, D. R., Chen, Q.; Using machine learning to identify novel neuroimaging phenotypes associated with cognitive dysfunction in Schizophrenia, 48th Annual Meeting of Society for Neuroscience, San Diego, CA, 2018. (Oral)

\item \textbf{Sadashivaiah, V.}, Goldman, A., Ulrich, B., Straub, R. E., Calliott, J. H., Breman, K. F., Mattay, V. S., Weinberger, D. R., Chen, Q.; Exploring Shared Brain Cognitive Networks and the Related Genetic Components using Three-way Parallel ICA, 73rd Annual Meeting of Society of Biological Psychiatry, New York, NY, 2018. (Poster) 

\item Chen, Q., Ursini, G., \textbf{Sadashivaiah, V.}, Radulescu, B., Straub, R. E., Breman, K. F., Mattay, V. S., Weinberger, D. R.,  Deciphering the association between polygenic risk for schizophrenia and hippocampal function, XXVth World Congress of Psychiatric Genetics, Orlando, FL, 2017. (Poster)

\item Ren, M., Chen, Q., \textbf{Sadashivaiah, V.}, Li, Y., Zhu, S., Mezeivtch, K., Hu, Z.,  Qin, LS L., Li, X., Tian, Q., Parades, D., Zhu, J., Wang, K. H., Weinberger, D. R., Yang, F., \textit{Abnormal hippocampal-mPFC synchrony in the KCNH2-3.1 transgenic mouse model}, 47th Annual Meeting of Society for Neuroscience, Washington D.C., 2017. (Poster)

\item \textbf{Sadashivaiah V}.,  Kyriakatos A.,  Zhang Y.,  Motta A.,  Auffret M.,  Petersen C. H.; Neural Circuits for goal-directed Sensorimotor Transformations, SRP and SUR Summer Research Symposium, EPFL School of Life Sciences, Lausanne, Switzerland, 2015. (Poster)

\item Pavan, K. R., Rao, S. A., Rao, V. V., Bongale, V. A., \textbf{Sadashivaiah, V.}; Real Time Non-Invasive Cardiac Health Monitoring System,  International Conference on Emergency Medical Service Systems - Innovation \& Entrepreneurship in Healthcare, AIIMS, New Delhi, India. October 2013. (Oral)
\end{presentations}
\end{rSection} 

\vspace{1em}

\begin{rSection}{Teaching Experience}

\begin{rSubsection}{Johns Hopkins University, \textit{Graduate Teaching Assistant}}{September 2015 - May 2017}{Models and Simulations, Statistical Mechanics and Thermodynamics, Systems Bioengineering III}{Baltimore, MD}
\item Assisted the Professors in designing and proof-reading assignments
\item Evaluated student assignments and quizzes, attended faculty led meetings
    \end{rSubsection}

\begin{rSubsection}{MIT Media Lab - Camera Culture Lab, \textit{Mentor}}{May 2014}{Kumbathon: Smart Cities Hackathon}{Cambridge, MA}
\item  Trained with a group of MIT undergraduates to build smart city related projects 
\item Traveled to Nasik, India to mentor projects at Kumbathon workshop
\item Assisted with signal processing and big data
    \end{rSubsection}

\begin{rSubsection}{Indian Institute of Technology Bombay, \textit{Mentor}}{January 2014}{Rethinking Engineering Design Execution (REDX) Hackathon}{Mumbai, India}
\item Guided $\sim$20 students in design and execution of innovative healthcare solutions
\item Projects involved for e.g., design of low cost, data driven ELISA system for rural imaging laboratories
\item Collaborated with doctors and researchers from MGH, Dana Faber Cancer Institute, Perkins Blind School etc.
    \end{rSubsection}

  \end{rSection}

\vspace{1em}

  \begin{rSection}{Leadership Experience}
          \begin{rSubsection}{Johns Hopkins University, \textit{Advocacy Chair}}{May 2016 -- May 2017}{Graduate Representative Organization}{Baltimore, MD}
\item Organized town halls every quarter with university administration to advocate graduate student needs and issues
\item Facilitated discussion of topics including student healthcare, maternity leave and dining options on campus
\item Assisted Social Chairs in organizing social and cultural events on campus
    \end{rSubsection}

\begin{rSubsection}{PES Institute of Technology, \textit{Core Team}}{May 2013 -- May 2015}{IEEE Student Branch}{Bengaluru, India}
\item Organized technical workshops for student community with invited speakers from industry and academia
\item Led a team of 5 to successfully organize a Spring Hackathon, ``Circuitus". Over 200 students participated
    \end{rSubsection}
  \end{rSection}   

\vspace{1em}

    \begin{rSection}{Skills}
\textbf{Programming:} Proficient in Python, MATLAB, R, \LaTeX, SQL, HTML, CSS, JavaScript, bash scripting \\
\textbf{Libraries:} TensorFlow, Keras, NEURON, SPM, git, OpenCV, Microsoft Office \\
\textbf{Data \& Models:} Deep Learning, Statistical Learning, Stochastic Modeling, Data Visualization, Big Data \\
\end{rSection} 

\begin{rSection}{Interests}
Rock Climbing, Taekwondo (ITF), Backpacking, Photography, Board Games, House Projects
\end{rSection} 


%\vspace{1em}
%
%    \begin{rSection}{Relevant Coursework}
%\begin{rSubsection}{Graduate Courses}{}{}{}
%\item Learning Theory, Topics in Systems Neuroscience, Models of a Neuron, Systems Bioengineering II\&III ({\it i.e.}Auditory Neurophysiology \& Models and Simulations), Digital Health and Biomedical Informatics, Introduction to Computational Medicine, Neuroscience of Pain
%\end{rSubsection}
%\begin{rSubsection}{Undergraduate Courses$^{*}$}{}{}{}
% \item Information Theory, Linear Algebra, Differential Equations, Calculus, Digital Signal Processing, Probability and Random Processes, Applied Mathematics, Signals and Systems, Computer Networks ($^{*}$Relevant courses only. See transcript for all courses)
%\end{rSubsection}
%\begin{rSubsection}{Online Coursework}{}{}{}
% \item Machine Learning, Deep Learning, Statistical Learning, Convex Optimization, Computational Neuroscience, Cellular mechanisms of Brain function, Algorithms
%\end{rSubsection}
%\end{rSection} 
%
%\vspace{1em}
%
%\begin{rSection}{References}
%References Furnished Upon Request
%\end{rSection}
\bibliographystyle{spmpsci}
\bibliography{CV_Vijay_Sadashivaiah}
\end{document}
